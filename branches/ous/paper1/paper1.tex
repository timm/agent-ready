% This is "sig-alternate.tex" V1.8 June 2007
% This file should be compiled with V2.3 of "sig-alternate.cls" June 2007
%
% This example file demonstrates the use of the 'sig-alternate.cls'
% V2.3 LaTeX2e document class file. It is for those submitting
% articles to ACM Conference Proceedings WHO DO NOT WISH TO
% STRICTLY ADHERE TO THE SIGS (PUBS-BOARD-ENDORSED) STYLE.
% The 'sig-alternate.cls' file will produce a similar-looking,
% albeit, 'tighter' paper resulting in, invariably, fewer pages.
%
% ----------------------------------------------------------------------------------------------------------------
% This .tex file (and associated .cls V2.3) produces:
%       1) The Permission Statement
%       2) The Conference (location) Info information
%       3) The Copyright Line with ACM data
%       4) NO page numbers
%
% as against the acm_proc_article-sp.cls file which
% DOES NOT produce 1) thru' 3) above.
%
% Using 'sig-alternate.cls' you have control, however, from within
% the source .tex file, over both the CopyrightYear
% (defaulted to 200X) and the ACM Copyright Data
% (defaulted to X-XXXXX-XX-X/XX/XX).
% e.g.
% \CopyrightYear{2007} will cause 2007 to appear in the copyright line.
% \crdata{0-12345-67-8/90/12} will cause 0-12345-67-8/90/12 to appear in the copyright line.
%
% ---------------------------------------------------------------------------------------------------------------
% This .tex source is an example which *does* use
% the .bib file (from which the .bbl file % is produced).
% REMEMBER HOWEVER: After having produced the .bbl file,
% and prior to final submission, you *NEED* to 'insert'
% your .bbl file into your source .tex file so as to provide
% ONE 'self-contained' source file.
%
% ================= IF YOU HAVE QUESTIONS =======================
% Questions regarding the SIGS styles, SIGS policies and
% procedures, Conferences etc. should be sent to
% Adrienne Griscti (griscti@acm.org)
%
% Technical questions _only_ to
% Gerald Murray (murray@acm.org)
% ===============================================================
%
% For tracking purposes - this is V1.8 - June 2007

\documentclass{sig-alternate}
\usepackage{times}
\usepackage{url}
\usepackage{cite}
\newcommand{\bi}{\begin{itemize}}
\newcommand{\ei}{\end{itemize}}
\newcommand{\be}{\begin{smallenum}}
\newcommand{\ee}{\end{smallenum}}
\newcommand{\tion}[1]{\S\ref{tion:#1}}
\newcommand{\eq}[1]{Equation~\ref{eq:#1}}
\newcommand{\fig}[1]{Figure~\ref{fig:#1}}
\newenvironment{smallitem}
 {\setlength{\topsep}{0pt}
  \setlength{\partopsep}{0pt}
  \setlength{\parskip}{0pt}
  \begin{itemize}
   \setlength{\leftmargin}{.2in}
  \setlength{\parsep}{0pt}
  \setlength{\parskip}{0pt}
  \setlength{\itemsep}{0pt}}
 {\end{itemize}}
\newenvironment{smallenum}
 {\setlength{\topsep}{0pt}
  \setlength{\partopsep}{0pt}
  \setlength{\parskip}{0pt}
  \begin{enumerate}
  \setlength{\leftmargin}{.2in}
  \setlength{\parsep}{0pt}
  \setlength{\parskip}{0pt}
  \setlength{\itemsep}{0pt}}
 {\end{enumerate}}
 
\begin{document}

\conferenceinfo{~}{~}

\title{An Agent Based Wedding Model}


\numberofauthors{1} 
\author{
\alignauthor
Oussama El-Rawas\\
       \affaddr{CPE,  West Virginia University,  USA}\\
       \email{oelrawas@mix.wvu.edu}
}

\maketitle
\begin{abstract}
In this course we have attempted to study agent based development and agent concepts to be applied in software development. In an effort to do just that, we have been presented with an architecture that aims to emulate the concept of agent programming. This Architecture's name is called Actory, and it is implemented in the Smalltalk scripting language. Using Actory, we attempt to model planning a wedding in an agent based manner. Though incomplete, we will present this implementation, along with the projected implementation and learning concepts that were planned, but not implemented. Before that, we will present Actory and how, in this author's opinion, Actory can be improved.
\end{abstract}

\keywords{Agents, BDI, Actory}

\section{Introduction}
Going into this course, one of our objectives was to learn and critique Agent-based programming methods in an attempt to understand further what agents are, or even what they ought to be.

\section{The {\secit Body} of The Paper}

\section{Conclusions}

\bibliographystyle{abbrv}
\bibliography{refs} 

\appendix

\end{document}
