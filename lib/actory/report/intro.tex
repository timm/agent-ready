\section{Introduction}
In recent years WVU and JPL have partnered together to study ways to improve software development processes. One example of such effort is the development and research of DDP models \cite{feather08}. A DDP model defines the relationship between requirements, risks, and risk mitigation strategies so they can be evaluated intelligently to reduce project costs while achieving the maximum requirements coverage. Generating these models requires a series of long meetings among the best project engineers at JPL to collect requirements, identify risks and discuss the cost and impact of mitigation strategies so the most requirements are achieved at the lowest cost. Finding the particular point where this goal is maximized is not a trivial task as the solution space grows exponentially based on the number of requirements, risks, and mitigations. It would be impossible for humans to search such vast space and therefore machines need to be used to test for the best solution. In some instances, when the models are not very big, it is possible to explore all possibilities and find the best solution that satisfies our goal but some other times this task becomes time-prohibitive and heuristic search based software engineering methods are required.